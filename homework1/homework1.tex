 %!TEX program = xelatex
\documentclass[a4paper]{article}

\usepackage{ctex}
\usepackage{fontspec, xunicode, xltxtra}
\usepackage{fancyhdr}
\pagestyle{fancy}
\usepackage{amsmath,amssymb}
\usepackage{lastpage}{}
\usepackage{inden tfirst}
\usepackage{listings}
\usepackage{xcolor}
\usepackage{listings}
%\usepackage{graphicx}

\setmainfont{Hiragino Sans GB}
\lhead{1500012775 钱昊}
\rhead{Page \thepage{} of \pageref{LastPage}}
\begin{document}
\section*{2.7}
\subsection*{(1)}
	用快速排序排序数组A,若A中存在主元素,则主元素必为A数组中位数($n$为奇数)或者两个最邻近中位数的平均值,令该值为$y$。

	然后遍历数组A,查看数组A中是否存在大于$\left \lfloor{\frac{n}{2}}\right \rfloor$次的$y$值。

	时间复杂度为$T(n)=O(n\log n)+O(n)=O(n\log n)$

\subsection*{(2)}
	同(1)找中位数的算法可以达到$O(n)$

	时间复杂度为$T(n)=O(n)+O(n)=O(n)$

\subsection*{(3)}
	芯片测试算法

	1.  如果n为偶数,将数组$A$两两分组。如果n为奇数,取出某个数$a$后,在当前数组上检验该元素是否为主元素,如果是,算法结束,如果不是再将,舍弃该数,再将数组$A$两两分组。

	2. 将每组数组的两个数组比较,若两个数相同,则随机将该组某个数放入数组$B$,否则则将其这两个数舍弃。

	3、 将数组$B$复制给数组$A$,如果数组$A$元素的个数小于等于$1$,则将剩下的唯一元素与原数组比较,看该元素是否为主元素。否则则继续转入1.

	该算法下有两个基本性质
	
	1.  若主元素存在,每一次迭代数组之后,主元素个数占总元素个数的比例仍然大雨$\frac{1}{2}$

	2.  每次迭代之后,如果算法需要继续,规模一定小于等于$\left \lfloor \frac{n}{2}\right \rfloor $ 迭代之前的数组规模为$n$

	证明:

	性质2显然成立。下面证明性质1。

	设数组$A$两两分组之后,情况如下:

		(1)  两个元素都为主元素,数量为$a$

		(2)  两个元素一个为主元素,一个不为主元素,数量为$b$

		(3)  两个元素均不为主元素,数量为$c$

		由主元素性质可以知道$2a+b > 2c+ b$,得到$a>c$,迭代一次之后,主元素的个数为$a$,总元素个数的最大值为$a+c$,所以性质2成立。

	算法时间复杂度:

	考虑最坏情况,每次迭代之后,子问题的规模最大为$\frac{n}{2}$,并且每次迭代都要扫描当前数组

	$$W(n)=W(\frac{n}{2})+O(n)$$

	由主定理可得 $W(n) = O(n)$

	最后还要扫描一遍原数组 $T(n) = W(n) +O(n) =O(n)$

	总的时间复杂度为$O(n)$

\section*{2.12}
1. 将集合B排序,生成数组L
2. 对A中每一个元素a,在L中用BinarySearch查找是否在L中存在和a一样的元素。

算法时间复杂度:
	集合B排序时间复杂度$O(m \log m)$,查找时间复杂度$O(n \log m)$
	总的时间复杂度$T(n)=O(m \log m) + O(n \log m) = O(n \log m) =O (n \log \log n)$

\section*{2.15}
\subsection*{(1)}
算法A 

$T(n) = \sum_{1}^{i}{n-i}=\frac{(n-1+n-i)*i}{2}=\frac{(2n-1-i)*i}{2}=O(n^{\frac{3}{2}})$

算法B

$T(n) = O(n \log n )$
\subsection*{(2)}

1. 令$d=n-i+1$

2. 采用Select算法,在数组S中采用,Select(S,k)

3. 此时S[n-i,n-1]为数组S中最大的i个数(并无有序)

4、 将S[n-i,n-1]采用快速排序,然后倒序输出。

时间复杂度:
	$T(n) = O(n)+O(i\log i) = O(n)$

\section*{2.18}

1. 遍历n个点,先算出每个点到原点$(0,0)$的距离,得到一个数组$A$

2. 用Select算法,运行Select(A,$\left \lfloor \sqrt{n} \right \rfloor$)

3. 此时A[0,$\left \lfloor \sqrt{n} \right \rfloor - 1$]为距离原点$(0,0)$最近的$\left \lfloor \sqrt{n} \right \rfloor$个点。

4. 对该区间进行快排,然后倒序输出。

时间复杂度:
$T(n)=O(n)+O(\left \lfloor \sqrt{n} \right \rfloor \log \left \lfloor \sqrt{n} \right \rfloor)=O(n)$

for $i \leftarrow 1$ to $n$

$d_i \leftarrow \sqrt{x_i^2+y_i^2}$,同时将每个点的下标与距离组成一个结构体,得到数组A

$k \leftarrow \left \lfloor \sqrt{n} \right \rfloor$

Select([$d_0,d_1,\cdots,d_{n-1}$],k)

Quicksort(A[0,$\left \lfloor \sqrt{n} \right \rfloor$-1])

for $i \leftarrow \left \lfloor \sqrt{n} \right \rfloor - 1$ to $0$

return A[i] 
\subsection*{2.23}

用Select算法找第$\left \lceil \frac{n}{4} \right \rceil$个小的数$a$和第$\left \lfloor \frac{3n}{4} \right \rfloor$个小的数$b$

if a == b:

 return None

else:

将数组分成A,B,C,D,E,其中A中元素小于$a$,B中元素等于$a$,C中元素大于$a$且小于$b$,D中元素等于$b$,E中元素大于$b$,若C非空,C中任意一个数都可以作为近似中值。

如果C为空,再特定判断a,b是否满足近似中值的条件。

时间复杂度

$T(n)=O(n)+O(n)=O(n)$

\subsection*{2.27}
对于n个商店,其坐标分别为$(x_1,y_1),(x_2,y_2),\cdots,(x_n,y_n)$,我们需要找到一个$(x,y)$

使得$min{|x-x_1|+|y-y_1|+|x-x_2|+|y-y_2|+\cdots+|x-x_n|+|y-y_n|}$

上式可化简为$min{|x-x_1|+\cdots|+|x-x_n|} + min{|y-y_1|+\cdots+|y-y_n|}$
$对于min{|x-x_1|+\cdots+|x-x_n|}$,我们只需找到序列$[x_1,x_2,\cdots,x_n]$的中位数即可

算法复杂度 

$T(n)=O(n)$
\end{document}