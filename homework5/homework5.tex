%!TEX program = xelatex
\documentclass{article}

\usepackage{ctex}
\usepackage{fontspec, xunicode, xltxtra}
\usepackage{fancyhdr}
\pagestyle{fancy}
\usepackage{amsmath,amssymb}
\usepackage{lastpage}{}
\usepackage{inden tfirst}
\usepackage{listings}
\usepackage{xcolor}
\usepackage{listings}
\usepackage{verbatim}
%\usepackage{graphicx}

\setmainfont{Hiragino Sans GB}
\lhead{1500012775 钱昊}
\rhead{Page \thepage{} of \pageref{LastPage}}
\begin{document}
\section*{4.2}
使用贪心法,把底面长度从小到大对物品排序,从标号小的物品开始依次装入库房,直到某个物品装不下为止。\\
证明命题:对任何输入,把物品按照从小到大的顺序装入得到最优解。\\
使用交换论证。\\
对于物品集合$S=\{1,2,\cdots,n\},假设A=\{i_1,i_2,i_3,\cdots,i_k\}是一个最优解,假设i_1=1,i_2=2\cdots,i_{j-1}=j-1,i_j \neq j,\\
则把i_j 替换成j,得到A^{'},则A^{'}也是一个可行解,也是一个最优解,至多再经过n-1次替换就可以得到B=\{1,2,3\cdots,k\},B也是一个最优解,命题得证$\\
时间复杂度$T(n)=O(n \log n)$
\section*{4.3}
算法思想:首先令$a[1]=d_1+4$,对$d_2,d_3,\cdots,d_n$检查,$\exists j, 使得d_j \le a[1] + 4 并且 d_{j+1} > a[1] + 4 $,则$a[2] = d_{j+1}+4$,直到包含所有的村庄 \\
伪代码:\\
a[1]<-d[1]+4 \\
k<-1 \\
for j<-2 to n  \\
\indent	if d[j]>a[k]+4 \\
\indent	then k<-k+1 \\
\indent \indent		a[k]<-d[j]+4 \\
return a \\
证明:采用归纳假设。\\
假设对于任何正整数k,存在最优解包含算法前k步选择的基站位置。\\
k=1,存在最优解包含a[1],若不然,存在另一个最优解b[1],$b[1]\neq a[1],并且b[1]<d_1+4=a[1]$,则将b[1]替换成a[1],仍是一个最优解,所以命题k等于1的时候成立。\\
假设对于k,存在最优解A包含算法前k步选择的基站位置\\
$A=\{a[1],a[2],\cdots,a[k]\} \cup B$ \\
假定$\{a[1],\cdots,a[k]\}$包含$\{d_1,\cdots,d_s\}$的村庄范围,则B的求解可以看成是对于$\{d_{s+1},\cdots,d_n\}$求解的子问题,根据归纳假设,B=$\{a[k+1],\cdots,a[k^{'}]\}$,从而证明了对于k+1时,命题依旧成立。从而整个命题得证。\\
时间复杂度$O(n)$
\section*{4.6}
采用贪心法:按照加工时间从小到大对作业进行排序,使得$t_i \le t_{i+1}$,按照标号从小到大的次序安排所有作业(并且没有空闲时间)\\
$time(i_j) = f(i_j)+t(i_j)=\sum_{k=1}^{j}t(i_k)$\\
f的总共完成时间为$time(f)=\sum_{k=1}^{n}time(i_k)=\sum_{j=1}^{n}t(i_j)(n-j+1)$
要使平均时间达到最小,即要使time(f)达到最小,由上述表达式可知,让序列t从小到大一定能达到最小,不然存在逆序$(i,j)$,交换i,j一定能达到更小的$time(f)$,经过有限次的变换,一定会达到没有逆序的序列。
\section*{4.17}
命题1:设T是图G的一棵最小生成树,C是G的一个圈,如果C中边的权彼此不等,那么在C中权最大的边不属于T。\\
证明:设e是C中权最大的边,假设e在T中,取C不在T的边e',$\omega(e')<\omega(e)$,用e‘替换e,得到一棵G的生成树T',且$\omega(T')<\omega(T)$,与T是最小生成树矛盾,命题得证。\\
命题2:对于$\forall e \in E,令s(e)=-w(e)是边的权值。$构造带权图G‘=<V,E,S>。G'的最小生成树T’满足本题所要求的最佳带宽性质。
命题2:$\forall u,v \in V,P'是在上述生成树T‘中连接u,v的唯一路径,\omega(P')=\omega(u,v)$
证明:反证法:,假若不是,那么G’中存在节点u,v,且有一条u-v的路径P,使得$\omega(P)>\omega(P')$,设e‘=((x,y)是T’中带宽最小的边。又由于$\omega(P')<\omega(P)$
\end{document}