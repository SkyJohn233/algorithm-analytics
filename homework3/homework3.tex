%!TEX program = xelatex
\documentclass[a4paper]{article}

\usepackage{ctex}
\usepackage{fontspec, xunicode, xltxtra}
\usepackage{fancyhdr}
\pagestyle{fancy}
\usepackage{amsmath,amssymb}
\usepackage{lastpage}{}
\usepackage{inden tfirst}
\usepackage{listings}
\usepackage{xcolor}
\usepackage{listings}
%\usepackage{graphicx}

\setmainfont{Hiragino Sans GB}
\lhead{1500012775 钱昊}
\rhead{Page \thepage{} of \pageref{LastPage}}
\begin{document}
\section*{3.3}
类似于0-1背包问题,问题转化为

\begin{align*}
  & max \sum_{i=1}^{n}v_i x_i (x_i 取 0 or 1) \\
  & \sum_{i=1}^{n}l_i x_i \le D
\end{align*}
令$C[k,y]$代表只考虑前k个货柜,库房长度为y的最大收益。

递推公式有

$$
C[k,y]=\left\{
\begin{aligned}
& C[k-1,y] (y<l_k) \\
& max(C[k-1,y],C[k-1,y-l_k]+v_k) D\ge y \ge l_k
\end{aligned}
\right.
$$
边界条件

$$
C[1,y] = \left \{
\begin{aligned}
v_1 \qquad y \ge l_1
0 \qquad  y < l_1
\end{aligned}
\right.
$$
伪代码:\\
for y<-1 to D \\
\hspace*{0.5cm}
	$C[1,y] <- v_1$ \\
for k<-2 to n \\
\hspace*{0.5cm}
for  y<-1 to D \\
\hspace*{1cm}
C[k,y] <- C[k-1,y] \\
\\hspace*{1cm}
i[k,y] <- i[k-1,y] (标记函数)\\
\hspace*{1cm}
if $y \ge l_k and C[k-1,y-l_k]+v_k >  C[k-1,y]$ \\
\hspace*{1cm}
then $C[k,y] \leftarrow C[k-1,y-l_k]+v_k$ \\
\hspace*{1.5cm}
$i[k,y]\leftarrow k$
最坏时间发杂度$O(nD)$

\section*{3.6}
令F(k,y)表示用前k种钱币,总钱数为y时可以付款。\\
F(k,y) = 0 or 1\\
递推方程为$F(k,y)=max(F(k-1,y),F(k-1,y-v_k))$ \\
边界条件:\\
$$
F(1,y) = \left\{
\begin{aligned}
& 1 \quad if \quad y==v_1 \\
& 0 \quad else
\end{aligned}
\right.
$$
$F(k,y) == 0 \quad if \quad y < 0$
标记函数则为i(k,y) \\
递推方程为\\
$$
i(k,y) =\left\{
\begin{aligned}
& k \quad if \quad F(k-1,y-v_k) == 1 \\
& i(k-1,y)  
\end{aligned}
\right.
$$

时间复杂度为$O(nM)$
\section*{3.12}
记顶点i,i+1,$\cdots$,j 构成的凸多边形记为A[i+1,j],原始问题为A[2,n]
记A[i,j]的最小权值为t[i,j],考虑edge(i-1,j),edge(i-1,j)一定属于凸多边形A[i,j]分成的某个子三角形ijk,则可以划分子问题为t(i,k)和t(k+1,j) \\
有递推方程
$$
t[i,j]=\left\{
\begin{aligned}
& 0 \quad i == j \\
& \min_{i\le k<j}(t(i,j),t(i,k)+t(k+1,j)+d(i-1,j)+d(i-1,k)+d(k,j))
\end{aligned}
\right.
$$
不考虑优化的话,时间复杂度为$O(n^3)$
\section*{3.15}
采用动态规划算法解决问题
记F[j],表示第1-j天加工任务的最大数目。\\
w[i,j]表示第i-j天连续工作的加工量。\\
则$w[i+1,j]=\sum_{k=i+1}^j{\min(x_k,s_{k-i})}$\\
w[i,j]满足递推方程
$$
\left\{
\begin{aligned}
& w[i+1,j]=w[i+1,j-1]+\min(x_j,s_{j-i}) \\
& w[i+1,i+1] =\min(x_{i+1},s_1)
\end{aligned}
\right.
$$
预处理w[]数组,可在$O(n^2)$时间内得到。

考虑F[j],假设最后一天检修为第i天(或者没检修)
有递推方程:\\
$$
F[j]=\max \left\{
\begin{aligned}
& \max_{0<i<j} (F[i-1]+w[i+1,j]) \quad j > 1 \\
& w[1,j] \quad j > 0
\end{aligned}
\right.
$$
特别地 F(0) = 0
时间复杂度为$O(n^2)$ \\
总时间复杂度为$O(n^2)$
\section*{3.17}
令F[i,j]表示从$a_{11}到a_{ij}$的路径上的数最小和。\\
则$F[i,j]=min(F[i-1,j],F[i-1,j-1])+a_{ij}$\\
特别地 $F[i,0] = +\infty, F[i,j] (j > i) = + \infty$ \\
问题即求$\min_{1\le j \le n}(F[n,j])$
确立标记函数pos(i,j)
递推方程为
$$
pos(i,j) = \left\{
\begin{aligned}
& j \quad if \quad F[i-1,j] <F[i-1,j-1] \\
& j - 1 \quad else
\end{aligned}
\right.
$$ 
时间复杂度$O(n^2)$
\end{document}