%!TEX program = xelatex
\documentclass[a4paper]{article}

\usepackage{ctex}
\usepackage{fontspec, xunicode, xltxtra}
\usepackage{fancyhdr}
\pagestyle{fancy}
\usepackage{amsmath,amssymb}
\usepackage{lastpage}{}
\usepackage{inden tfirst}
\usepackage{listings}
\usepackage{xcolor}
\usepackage{listings}
%\usepackage{graphicx}

\setmainfont{Hiragino Sans GB}
\lhead{1500012775 钱昊}
\rhead{Page \thepage{} of \pageref{LastPage}}
\begin{document}
\section*{5.2}
按照价格从小到大对来自不同供应商的零件排序。\\
假设解向量为$<x_1,x_2,\cdots,x_n>,x_i =j 表示第i号零件由j号供应商供货。$\\
结点$<x_1,x_2,\cdots,x_n>$表示已经选择了前k号零件的供应商。现在处理第k+1号零件。 \\
约束条件:选择下一个零件后总价格不超过120。
代价函数:\\
$F=\sum_{i=1}^k \omega_{ix_i}+\sum_{j=k+1}^{n}min{\omega (w_{jl})}$

\section*{5.5}
解向量为$<x_1,x_2,\cdots,x_8$ \\
在结点$<x_1,x_2,\cdots,x_k>处$,下一个结点条件是$x_{k+1}与x_1,x_2,\cdots,x_k$不在同一列,不在同一行,也不在同一个对角线。然后按广度优先顺序遍历这课树。对于n后问题,最坏情况下时间复杂度为$O(n^n)$

\section*{5.6}
解向量为$<x_1,x_2,\cdots,x_n>$ \\
$x_i=1$表示数$n_i$在子集中。\\
在节点$<x_1,x_2,\cdots,x_k>$,约束条件为$B(i)=\sum_{i=1}^k a_i x_i  < M$

\section*{5.8}
(1) 上述电路板问题的实例,该实例的最优解之一是${2,1,3,6,4,5,7,8}$,排列密度为2。\\
(2) 搜索空间是排列树,对于某个节点$<x_1,x_2,\cdots,x_i$,选择下一步节点中的$x_{i+1} \in \{1,2,3,\cdots,n\} - B$为约束条件。界时目前得到的最小的排列密度。 \\
令total[j]表示连接块j所连接的电路板总数,now[j]表示$\{x_1,x_2,\cdots,x_i\}$中已经包含在$N_j$中的电路板数。\\
则$N_j$跨越第i和i+1插槽的条件等价于 now[j] < total[j] 并且 now[j] > 0 \\
观察这组数据 $N_2,N_3,N_4$ 有共同的电路板3,所以排列密度有一个下界2,所以当存在电路板排列,使得排列密度为2时,该排列为该电路板问题实例的最优解。
\end{document}